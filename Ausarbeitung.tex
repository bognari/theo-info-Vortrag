\documentclass[a4paper]{article}

\usepackage[utf8x]{inputenc}
\PrerenderUnicode{äöüÄÖÜß}
\usepackage{amsmath}
\usepackage{amsthm}

\newtheorem{lemma}{Lemma}

\begin{document}
\begin{lemma}
	Für jeden $U_2$ Schaltkreis $\beta$, gibt es einen äquivalnten Schaltkreis, welcher höchstens
	2 mal so groß ist wie $\beta$, in dem nur NOT-Gatter nur bei den Variablen benutzt werden.
\end{lemma}

Bei einem $U_2$ Schaltkreis handelt es sich um einen Schaltkreis bei dem nur die Gatter AND, OR und NOT mit 2 Eingängen verwendet werden.\\

Als erstes werden die Gatter topologisch sortiert. Also Gatter die andere Gatter an ihren Eingängen haben kommen nach diesen.\\
Nun beginnen wir mit dem obersten Gatter. Sollte dessen Ausgang verneint sein, so wenden wir darauf die Regel von deMorgan\footnote{$\overline{a\land{}b}=\overline{a}\lor\overline{b}$ Analog für OR} an.\\
Sollten die Ausgänge verneint sein so verschieben wir diese Verneinung auf die Ausgänge der vorhergehenden Gatter, wobei sich doppelte Verneinungen Aufheben.\\
Dies wird wiederholt für alle vorhergehenden Gatter.\\
Für den Fall, dass ein Gatter negiert und nicht negiert verwendet wird, wird dieses Gatter verdoppelt.\\
\qed{}
\end{document}